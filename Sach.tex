\documentclass[12pt]{scrartcl}  % hoặc memoir
\usepackage[utf8]{inputenc}
\usepackage[T5]{fontenc}
\usepackage[vietnamese]{babel}
\usepackage{listings}
\usepackage{fancyhdr}
\usepackage{graphicx}
\usepackage{xcolor}
\usepackage{tcolorbox}
\usepackage{tikz}
\usetikzlibrary{calc}
\usepackage{hyperref}
\usepackage{geometry}
\usepackage{minted}
\geometry{a4paper, margin=2.5cm}

% Thiết lập code C++
\definecolor{mybg}{rgb}{0.97,0.97,0.97}
\definecolor{mykeyword}{rgb}{0.13,0.13,0.7}
\definecolor{mycomment}{rgb}{0.0,0.5,0.0}
\definecolor{mystring}{rgb}{0.7,0.13,0.13}

\lstset{
  language=C++,
  backgroundcolor=\color{mybg},
  basicstyle=\ttfamily\small,
  keywordstyle=\color{mykeyword}\bfseries,
  commentstyle=\itshape\color{mycomment},
  stringstyle=\color{mystring},
  numbers=left,
  numberstyle=\tiny,
  stepnumber=1,
  frame=single,
  breaklines=true,
  captionpos=b,
  showspaces=false 
}

\pagestyle{fancy}
\fancyhf{}
\fancyhead[L]{Tài liệu C++}
\fancyhead[R]{\thepage}

\title{Hướng dẫn học lập trình C++ cơ bản đến nâng cao}
\author{Dinh Xuan Minh}
\date{\today}

\begin{document}
\maketitle
\tableofcontents

\section*{Lời nói đầu}
Cuốn sách này tổng hợp các đề thi và lời giải chi tiết các bài lập trình dành cho học sinh giỏi lớp 9. Mỗi bài gồm phần đề, phân tích, hướng dẫn giải và code mẫu.

\section{ĐỀ HSG 9 THCS TỈNH HẢI DƯƠNG NĂM HỌC 2024-2025}
\textbf{Thời gian làm bài: 150 phút}

\textbf{Độ khó: }

\subsection{Bài 1: Số tam giác}
\textbf{Đề bài:}
Một lục giác đều với độ dài cạnh là số nguyên dương có thể được ghép bằng một số tam giác đều có độ dài cạnh bằng 1. Ví dụ dưới đây là hai hình lục giác đều được ghép bằng các tam giác đều cạnh 1.
\begin{center}
\begin{tikzpicture}[scale=1]
    \foreach \i in {0,...,5} {
        \coordinate (A\i) at (60*\i:1);
    }
    \draw[thick] (A0) -- (A1) -- (A2) -- (A3) -- (A4) -- (A5) -- cycle;
    \foreach \i in {0,...,5} {
        \draw[thick] (0,0) -- (A\i);
    }
    \node at (0,-1.5) {Hình 1};
    \node at (0,-2.2) {\small Hình lục giác đều độ dài cạnh 1};
    \node at (0,-2.7) {\small được ghép bằng 6 tam giác đều độ dài cạnh 1};
\end{tikzpicture}
\hspace{2cm}
\begin{tikzpicture}[scale=0.8]
    \def\sidelength{1}
    \pgfmathsetmacro{\triheight}{sqrt(3)/2 * \sidelength}
    \draw[thick] (-2,0) -- (-1, {sqrt(3)}) -- (1, {sqrt(3)}) -- (2,0) -- (1, {-sqrt(3)}) -- (-1, {-sqrt(3)}) -- cycle;
    \foreach \i in {-2,-1,0,1,2} {
        \pgfmathsetmacro{\currenty}{\i * \triheight} % Calculate y-coordinate
        \pgfmathsetmacro{\xlim}{2 - abs(\i) * 0.5} % Calculate x-limit based on distance from center (abs(i))
        \draw[thick] (-\xlim, \currenty) -- (\xlim, \currenty);
    }
    \coordinate (O) at (0,0);
    \coordinate (V0) at (2,0);
    \coordinate (V1) at (1, {sqrt(3)});
    \coordinate (V2) at (-1, {sqrt(3)});
    \coordinate (V3) at (-2,0);
    \coordinate (V4) at (-1, {-sqrt(3)});
    \coordinate (V5) at (1, {-sqrt(3)});
    \draw[thick] (O) -- (V0);
    \draw[thick] (O) -- (V1);
    \draw[thick] (O) -- (V2);
    \draw[thick] (O) -- (V3);
    \draw[thick] (O) -- (V4);
    \draw[thick] (O) -- (V5);
    \coordinate (I0) at (1,0);
    \coordinate (I1) at (0.5, {sqrt(3)/2});
    \coordinate (I2) at (-0.5, {sqrt(3)/2});
    \coordinate (I3) at (-1,0);
    \coordinate (I4) at (-0.5, {-sqrt(3)/2});
    \coordinate (I5) at (0.5, {-sqrt(3)/2});
    \draw[thick] (I0)--(I1)--(I2)--(I3)--(I4)--(I5)--cycle;
    \foreach \i in {0,...,5} {
        \pgfmathsetmacro{\j}{mod(\i+1,6)} 
        \coordinate (MidOuter) at ($(V\i)!0.5!(V\j)$);
        \coordinate (MidRadial_i) at (I\i);
        \coordinate (MidRadial_j) at (I\j);
        \draw[thick] (MidOuter) -- (MidRadial_i);
        \draw[thick] (MidOuter) -- (MidRadial_j);
    }
    \node at (0,-2.2) {\small Hình lục giác đều độ dài cạnh 2 được ghép bằng};
    \node at (0,-2.7) {\small 24 tam giác đều độ dài cạnh 1};
\end{tikzpicture}

\end{center}
\textbf{Yêu cầu:}
Hỏi rằng số tam giác đều tối thiểu là bao nhiêu để ghép được $n$ hình lục giác đều với độ dài các cạnh lần lượt là $1, 2, ..., n$.

\textbf{Dữ liệu vào:}

Gồm $T + 1$ dòng:
\begin{itemize}
    \item Dòng đầu tiên chứa số nguyên dương $T$ ($1 \leq T \leq 10^5$) là số lượng test.
    \item Tiếp theo là $T$ dòng, mỗi dòng tiếp theo chứa một số nguyên dương $n$ ($1 \leq n \leq 10^6$) mô tả một bộ dữ liệu.
\end{itemize}

\textbf{Dữ liệu ra:}
Gồm $T$ dòng, mỗi dòng in ra số tam giác đều tối thiểu cần thiết cho bộ dữ liệu tương ứng.

\textbf{Ràng buộc dữ liệu:}

\begin{itemize}
    \item Subtask 1 (60 điểm): $T \le 10$, $n \leq 1000$.
    \item Subtask 2 (40 điểm): $10 < T \le 10^5 $.
\end{itemize}


\textbf{Ví dụ:}
\begin{tcolorbox}[colback=gray!5!white, colframe=blue!50!black, title=Input]
3\\
1\\
2\\
3
\end{tcolorbox}
\begin{tcolorbox}[colback=gray!5!white, colframe=green!50!black, title=Output]
6\\
30\\
84
\end{tcolorbox}

\textbf{Hướng dẫn giải:}


\textbf{Solution C++:}
\begin{lstlisting}
#include <bits/stdc++.h>
using namespace std;
int main() {
    int T;
    cin >> T;
    while (T--) {
        int n;
        cin >> n;
        long long result = 3 * n * (n + 1);
        cout << result << endl;
    }
    return 0;
}
\end{lstlisting}

\subsection{Bài 2: Chia phần}
\textbf{Đề bài:}
Cho dãy số nguyên dương $a_1, a_2, ..., a_n$. Chia dãy này thành hai phần
\begin{itemize}
    \item Phần thứ nhất gồm các số $a_1, a_2, ..., a_k$.
    \item Phần thứ hai gồm các số còn lại.
\end{itemize}

\textbf{Yêu cầu:}
Gọi $T_1$ và $T_2$ lần lượt là tổng các số trong phần thứ nhất và phần thứ hai. Hãy tìm giá trị nhỏ nhất của $|T_1 - T_2|$.
\textbf{Dữ liệu vào:}

Gồm hai dòng:
\begin{itemize}
    \item Dòng đầu tiên chứa số nguyên dương $n (2 < n \leq 10^6)$ là số lượng phần tử trong dãy.
    \item Dòng thứ hai chứa $n$ số nguyên dương $a_1, a_2, ..., a_n$ ($ | a_i | \le 10^9, \forall i = 1,2,...,n$) cách nhau bởi dấu cách.
\end{itemize}

\textbf{Dữ liệu ra:}
Gồm một dòng duy nhất chứa giá trị nhỏ nhất của $|T_1 - T_2|$.

\textbf{Ràng buộc dữ liệu:}

\begin{itemize}
    \item Subtask 1 (75 điểm): $n \le 5000$.
    \item Subtask 2 (25 điểm): $n > 5000$.
\end{itemize}

\textbf{Ví dụ:}
\begin{tcolorbox}[colback=gray!5!white, colframe=blue!50!black, title=Input]
5\\
1 2 3 4 5
\end{tcolorbox}
\begin{tcolorbox}[colback=gray!5!white, colframe=green!50!black, title=Output]
3
\end{tcolorbox}

\textbf{Hướng dẫn giải:}
Dùng toán tử điều kiện hoặc hàm so sánh.

\textbf{Solution C++:}
\begin{lstlisting}
#include <bits/stdc++.h>
using namespace std;
int main() {
    int n;
    cin >> n;
    long long a[n];
    long long sum = 0;
    for (int i = 0; i < n; i++) {
        cin >> a[i];
        sum += a[i];
    }
    
    long long min_diff = LLONG_MAX;
    long long T1 = 0;
    
    for (int i = 0; i < n - 1; i++) {
        T1 += a[i];
        long long T2 = sum - T1;
        min_diff = min(min_diff, abs(T1 - T2));
    }
    
    cout << min_diff << endl;
    return 0;
}
\end{lstlisting}

\subsection{Bài 3: Kiểm tra số chẵn lẻ}
\textbf{Đề bài:}
An có $n$ đoạn thẳng. Cậu ta nhận thấy rằng một số đoạn thẳng cùng chiều dài nên có thể xếp thành những hình vuông.

\textbf{Yêu cầu:}

Hỏi rằng số hình vuông nhiều nhất An có thể xếp được là bao nhiêu?
\textbf{Dữ liệu vào:}
Gồm hai dòng:
\begin{itemize}
    \item Dòng đầu tiên chứa số nguyên dương $n$ ($1 \leq n \leq 3 \times 10^5$) là số lượng đoạn thẳng.
    \item Dòng thứ hai chứa $n$ số nguyên dương $a_1, a_2, ..., a_n$ ($1 \leq a_i \leq 10^{18}$) là độ dài của các đoạn thẳng.
\end{itemize}
\textbf{Dữ liệu ra:}
Gồm một số nguyên duy nhất là số lượng hình vuông nhiều nhất An có thể xếp được.
\textbf{Ràng buộc dữ liệu:}

\begin{itemize}
    \item Subtask 1 (30\%): $n \leq 2000, a_i \leq 10^6$.
    \item Subtask 2 (30\%): $n > 2000, a_i \leq 10^6$.
    \item Subtask 3 (40\%): $n > 2000, a_i \leq 10^{18}$.
\end{itemize}


\textbf{Ví dụ:}
\begin{tcolorbox}[colback=gray!5!white, colframe=blue!50!black, title=Input]
9\\
2 2 2 9 2 2 2 2 2
\end{tcolorbox}
\begin{tcolorbox}[colback=gray!5!white, colframe=green!50!black, title=Output]
2
\end{tcolorbox}

\textbf{Hướng dẫn giải:}
Kiểm tra $n$ chia hết cho 2 hay không.

\textbf{Solution C++:}
\begin{lstlisting}
#include <bits/stdc++.h>
using namespace std;
int main() {
    int n;
    cin >> n;
    long long a[n];
    for (int i = 0; i < n; i++) {
        cin >> a[i];
    }
    
    map<long long, int> count_map;
    for (int i = 0; i < n; i++) {
        count_map[a[i]]++;
    }
    
    long long squares = 0;
    for (auto& pair : count_map) {
        squares += pair.second / 4;
    }
    
    cout << squares << endl;
    
    return 0;
}
\end{lstlisting}

\subsection{Bài 4: Tích lớn nhất}
\textbf{Đề bài:}
Cho một dãy số nguyên dương $a_1, a_2, ..., a_n$.

\textbf{Yêu cầu:}
Hãy tính giá trị lớn nhất của biểu thức $a_i \times a_j \times a_k$ với $1 \leq i < j < k \leq n$.

\textbf{Dữ liệu vào:}
Gồm hai dòng:
\begin{itemize}
    \item Dòng đầu tiên chứa số nguyên dương $n$ ($3 \leq n \leq 3 \times 10^5$) là số lượng phần tử trong dãy.
    \item Dòng thứ hai chứa $n$ số nguyên dương $a_1, a_2, ..., a_n$ ($ |a_i| \leq 10^6$) cách nhau bởi dấu cách.
\end{itemize}

\textbf{Dữ liệu ra:}
Gồm một số nguyên duy nhất là giá trị lớn nhất của biểu thức $a_i \times a_j \times a_k$.

\textbf{Ràng buộc dữ liệu:}
\begin{itemize}
    \item Subtask 1 (40\%): $n \leq 100$.
    \item Subtask 2 (60\%): $n > 100$.
\end{itemize}
\textbf{Ví dụ:}
\begin{tcolorbox}[colback=gray!5!white, colframe=blue!50!black, title=Input]
6\\
5 2 10 1 3 2
\end{tcolorbox}
\begin{tcolorbox}[colback=gray!5!white, colframe=green!50!black, title=Output]
150
\end{tcolorbox}
\textbf{Hướng dẫn giải:}

\textbf{Solution C++:}

\begin{lstlisting}
#include <iostream>
using namespace std;

int main() {
    int n;
    cin >> n;
    long long max_product = 0;
    long long a[3];
    
    for (int i = 0; i < n; i++) {
        long long x;
        cin >> x;
        if (i < 3) {
            a[i] = x;
        } else {
            if (x > a[0]) {
                a[0] = x;
            }
            if (x > a[1]) {
                a[1] = a[0];
                a[0] = x;
            } else if (x > a[2]) {
                a[2] = x;
            }
        }
    }
    
    max_product = a[0] * a[1] * a[2];
    cout << max_product << endl;
    
    return 0;
}
\end{lstlisting}

\subsection{Bài 5: Đếm số cặp}
\textbf{Đề bài:}
Cho một dãy số nguyên dương $a_1, a_2, ..., a_n$.
\textbf{Yêu cầu:}
Hãy đếm số cặp $(i, j)$ với $1 \leq i < j \leq n$ thỏa mãn tính chất: Số $a_i \times a_j$ là một số chính phương (số nguyên dương $x$ được gọi là chính phương nếu tồn tại một số nguyên dương $y$ sao cho $x = y^2$).
\textbf{Dữ liệu vào:}
Gồm hai dòng:
\begin{itemize}
    \item Dòng đầu tiên chứa số nguyên dương $n$ ($n \leq 10^6$) là số lượng phần tử trong dãy.
    \item Dòng thứ hai chứa $n$ số nguyên dương $a_1, a_2, ..., a_n$ ($1 \leq a_i \leq 10^6, \forall i = 1,2,...,n$) cách nhau bởi dấu cách.
\end{itemize}
\textbf{Dữ liệu ra:}
Gồm một số nguyên duy nhất là số lượng cặp $(i, j)$ thỏa mãn tính chất đã nêu.
\textbf{Ràng buộc dữ liệu:}
\begin{itemize}
    \item Subtask 1 (40\%): $n \leq 2000$.
    \item Subtask 2 (40\%): $n > 2000, a_i \leq 10^4$.
    \item Subtask 3 (20\%): $n > 2000, a_i \leq 10^6$.
\end{itemize}
\textbf{Ví dụ:}
\begin{tcolorbox}[colback=gray!5!white, colframe=blue!50!black, title=Input]
5\\
2 8 3 75 27
\end{tcolorbox}
\begin{tcolorbox}[colback=gray!5!white, colframe=green!50!black, title=Output]
4  
\end{tcolorbox}
\textbf{Hướng dẫn giải:}
\textbf{Solution C++:}
\begin{lstlisting}
#include <bits/stdc++.h>
using namespace std;
int main() {
    int n;
    cin >> n;
    vector<int> a(n);
    for (int i = 0; i < n; i++) {
        cin >> a[i];
    }
    
    unordered_map<int, int> freq;
    for (int i = 0; i < n; i++) {
        freq[a[i]]++;
    }
    
    long long count = 0;
    for (auto& pair1 : freq) {
        for (auto& pair2 : freq) {
            if (pair1.first <= pair2.first) {
                int product = pair1.first * pair2.first;
                int root = sqrt(product);
                if (root * root == product) {
                    if (pair1.first == pair2.first) {
                        count += (long long)pair1.second * (pair1.second - 1) / 2;
                    } else {
                        count += (long long)pair1.second * pair2.second;
                    }
                }
            }
        }
    }
    
    cout << count << endl;
    
    return 0;
}
\end{lstlisting}

\section{ĐỀ HSG 9 THCS TỈNH ĐIỆN BIÊN NĂM HỌC 2024-2025}
\textbf{Thời gian làm bài: 150 phút}
\textbf{Độ khó: }
\subsection{Bài 1: Tính tiền}
\textbf{Đề bài:}
Trong đợt Hội chợ thương mai Điện Biên năm 2024. Để kích cầu một doanh nghiệp đã đưa ra chương trình khuyến mãi. Theo đó, nếu tổng giá trị hóa đơn lớn hơn hoặc bằng 2,000,000 đồng, khách hàng sẽ được giảm giá 15\% trên tổng giá trị hóa đơn. Nếu không đạt điều kiện trên, khách hàng sẽ phải thanh toán toàn bộ giá trị hóa đơn mà không được giảm giá.

\textbf{Yêu cầu:}

Viết chương trình để tính số tiền thực tế khách hàng phải thanh toán dựa trên: Số lượng hàng bán (ký hiệu là $a$), đơn giá của mỗi mặt hàng (ký hiệu là $b$ đồng).

\textbf{Dữ liệu vào:}

Gồm một dòng duy nhất chứa hai số nguyên dương $a$ và $b$ ($1 \leq a \leq 10^4,2 \times 10 \leq b \leq 10^9$) cách nhau bởi dấu cách.

\textbf{Dữ liệu ra:}

Gồm một dòng duy nhất số tiền thực tế khách hàng cần thanh toán.Kết quả đảm bảo là số nguyên.

\textbf{Ràng buộc dữ liệu:}
\begin{itemize}
    \item Subtask 1 (80\%): $1 \leq a \leq 100$, $2 \times 10 \leq b \leq 10^6$.
    \item Subtask 2 (20\%): $10^2 < a \leq 10^4$, $10^6 < b \leq 10^9$.
\end{itemize}

\textbf{Ví dụ:}
\begin{tcolorbox}[colback=gray!5!white, colframe=blue!50!black, title=Input]
2 1000000
\end{tcolorbox}
\begin{tcolorbox}[colback=gray!5!white, colframe=green!50!black, title=Output]
1700000
\end{tcolorbox}
\textbf{Hướng dẫn giải:}
\textbf{Solution C++:}
\begin{lstlisting}
#include <bits/stdc++.h>
using namespace std;
int main() {
    long long a, b;
    cin >> a >> b;
    long long total = a * b;
    if (total >= 2000000) {
        total -= total * 15 / 100; 
    }
    cout << total << endl;
    return 0;
}
\end{lstlisting}

\subsection{Bài 2: Số cùng nhau}
\textbf{Đề bài:}
Trong một ngôi làng cổ, các số tự nhiên thường kết đôi để cùng nhau thực hiện những nhiệm vụ đặc biệt. Nhưng không phải cặp số nào cũng có thể đồng hành, chỉ những cặp "tương thích hoàn hảo" mới được chọn. Một cặp số $i$ và $j$ được xem là tương thích hoàn hảo nếu chúng không có bất kỳ ước chung nào khác ngoài số $1$, nghĩa là $\mathrm{UCLN}(i, j) = 1$.

\textbf{Yêu cầu:}

Nhiệm vụ của bạn là giúp trưởng làng đếm xem có bao nhiêu cặp số cùng nhau trong đoạn $[a, b]$ được chọn làm bạn đồng hành lý tưởng.

\textbf{Dữ liệu vào:}

Một dòng duy nhất chứa số nguyên dương $a$ và $b$, cách nhau bởi dấu cách ($1 \leq a < b \leq 10^3$).

\textbf{Dữ liệu ra:}

In ra một số nguyên là số lượng cặp số $(i, j)$ thỏa mãn điều kiện trên.

\textbf{Ràng buộc dữ liệu:}

\begin{itemize}
    \item Subtask 1 (40\%): $1 \leq a, b \leq 10$.
    \item Subtask 2 (30\%): $10 < a, b \leq 100$.
    \item Subtask 3 (30\%): $100 < a, b \leq 1000$.
\end{itemize}

\textbf{Ví dụ:}
\begin{tcolorbox}[colback=gray!5!white, colframe=blue!50!black, title=Input]
1 5
\end{tcolorbox}
\begin{tcolorbox}[colback=gray!5!white, colframe=green!50!black, title=Output]
9
\end{tcolorbox}

\textbf{Giải thích:}
Trong đoạn $[1, 5]$, các cặp số cùng nhau là: $(1, 2)$, $(1, 3)$, $(1, 4)$, $(1, 5)$, $(2, 3)$, $(2, 5)$, $(3, 4)$, $(3, 5)$ và $(4, 5)$. Tổng cộng có $9$ cặp.

\textbf{Hướng dẫn giải:}

Sử dụng hàm Euler $\varphi(k)$ để đếm số lượng số nguyên nhỏ hơn $k$ và nguyên tố cùng nhau với $k$.

\textbf{Solution C++:}
\begin{lstlisting}
#include <bits/stdc++.h>
using namespace std;
const int N = 1e6 + 5;
int phi[N];
int main() {
    int n;
    cin >> n;
    for (int i = 1; i <= n; i++) phi[i] = i;
    for (int i = 2; i <= n; i++) {
        if (phi[i] == i) {
            for (int j = i; j <= n; j += i)
                phi[j] -= phi[j] / i;
        }
    }
    long long res = 0;
    for (int i = 2; i <= n; i++) res += phi[i];
    cout << res << endl;
    return 0;
}
\end{lstlisting}

\subsection{Bài 3: Thu nhập}
\textbf{Đề bài:}
Để biết được mức thu nhập trung bình hàng tháng của các hộ dân trong thành phố. Thành phố đã tiến hành khảo sát $N$ hộ dân, hộ dân thứ $i$ có thu nhập $a_i$ triệu đồng. Nhìn vào số liệu thống kê nhận thấy rằng mỗi hộ dân có thu nhập khác nhau, lãnh đạo thành phố muốn biết mức thu nhập thấp nhất và cao nhất cũng như mức thu nhập nào phổ biến trong các hộ dân nhất để thành phố có kế hoạch phát triển kinh tế giàu mạnh.

\textbf{Yêu cầu:}
Hãy cho biết thu nhập thấp nhất, cao nhất và đếm các hộ dân có thu nhập phổ biến nhất trong thành phố.

\textbf{Dữ liệu vào:}
Gồm hai dòng:
\begin{itemize}
    \item Dòng đầu tiên chứa số nguyên dương $N$ ($1 \leq N \leq 10^5$) là số lượng hộ dân.
    \item Dòng thứ hai chứa $N$ số nguyên dương $a_i$ ($1 \leq a_i \leq 10^9, i = 1 ... N$) là thu nhập của các hộ dân, cách nhau bởi dấu cách.
\end{itemize}

\textbf{Dữ liệu ra:}
Gồm hai dòng:

\begin{itemize}
    \item Dòng đầu tiên in ra thu nhập thấp nhất và cao nhất, cách nhau bởi dấu cách.
    \item Dòng thứ hai in ra số nguyên dương là số hộ dân nhiều nhất có mức thu nhập bằng nhau.
\end{itemize}

\textbf{Ràng buộc dữ liệu:}
\begin{itemize}
    \item Subtask 1 (40\%): $1 \leq N \leq 10^2$.
    \item Subtask 2 (30\%): $10^2 < N \leq 10^3$.
    \item Subtask 3 (30\%): $10^3 < N \leq 10^5$.
\end{itemize}

\textbf{Ví dụ:}
\begin{tcolorbox}[colback=gray!5!white, colframe=blue!50!black, title=Input]
9\\
5 1 5 8 6 2 3 6 3
\end{tcolorbox}
\begin{tcolorbox}[colback=gray!5!white, colframe=green!50!black, title=Output]
1 8\\
2
\end{tcolorbox}
\textbf{Hướng dẫn giải:}
\textbf{Solution C++:}
\begin{lstlisting}
#include <bits/stdc++.h>
using namespace std;
int main() {
    int n;
    cin >> n;
    vector<int> a(n);
    for (int i = 0; i < n; i++) {
        cin >> a[i];
    }
    
    int min_income = *min_element(a.begin(), a.end());
    int max_income = *max_element(a.begin(), a.end());
    
    unordered_map<int, int> freq;
    for (int i = 0; i < n; i++) {
        freq[a[i]]++;
    }
    
    int max_count = 0;
    for (auto& pair : freq) {
        max_count = max(max_count, pair.second);
    }
    
    cout << min_income << " " << max_income << endl;
    cout << max_count << endl;
    
    return 0;
}
\end{lstlisting}

\subsection{Bài 4: Đếm dãy con liên tiếp}
\textbf{Đề bài:}

Cho dãy số $A$ có $n$ số nguyên $a_1, a_2, \ldots, a_n$. Một dãy con liên tiếp các số hạng của dãy $A$ là dãy các số hạng từ số hạng $a_i$ đến số hạng $a_j$ ($1 \leq i \leq j \leq n$).

\textbf{Yêu cầu:}

Hãy cho biết dãy $A$ có bao nhiêu dãy con liên tiếp mà giá trị tuyệt đối của tổng các số hạng trong dãy con đó lớn hơn một số nguyên dương $S$ cho trước.

\textbf{Dữ liệu vào:}

Gồm hai dòng:
\begin{itemize}
    \item Dòng thứ nhất chứa hai số nguyên dương $n$ và $S$ ($n \leq 10^5, S \leq 10^{14}$);
    \item Dòng thứ hai chứa $n$ số nguyên $a_1, a_2, \ldots, a_n$ ($|a_i| \leq 10^9$). Hai số liên tiếp trên cùng dòng được ghi cách nhau bởi dấu cách.
\end{itemize}

\textbf{Dữ liệu ra:}

Gồm một số nguyên duy nhất là số dãy con liên tiếp thỏa mãn yêu cầu của bài toán.

\textbf{Ràng buộc dữ liệu:}
\begin{itemize}
    \item Subtask 1 (50\%): $n \leq 100$.
    \item Subtask 2 (30\%): $n \leq 10^3$.
    \item Subtask 3 (20\%): $n \leq 10^5$.
\end{itemize}

\textbf{Ví dụ 1:}
\begin{tcolorbox}[colback=gray!5!white, colframe=blue!50!black, title=Input]
4 4\\
5 - 1 8 -5
\end{tcolorbox}
\begin{tcolorbox}[colback=gray!5!white, colframe=green!50!black, title=Output]
6
\end{tcolorbox}

\textbf{Giải thích:}
Các dãy con liên tiếp có tổng tuyệt đối lớn hơn $4$ là: $$\left\{5 \right\}, \left\{ 8 \right\}, \left\{ -1, 8 \right\}, \left\{ 5, -1, 8 \right\}, \left\{ 5, -1, 8, -5 \right\}$$.

\textbf{Ví dụ 2:}
\begin{tcolorbox}[colback=gray!5!white, colframe=blue!50!black, title=Input]
10 7\\
-4 9 2 -11 -3 8 -6 5 -3 1
\end{tcolorbox}
\begin{tcolorbox}[colback=gray!5!white, colframe=green!50!black, title=Output]
12
\end{tcolorbox}

\textbf{Hướng dẫn giải:}

Sử dụng prefix sum và hai con trỏ để đếm số lượng dãy con liên tiếp có tổng tuyệt đối lớn hơn $S$.

\textbf{Solution C++:}
\begin{lstlisting}
#include <bits/stdc++.h>
using namespace std;
typedef long long ll;
int main() {
    int n;
    ll S;
    cin >> n >> S;
    vector<ll> a(n+1, 0), prefix(n+1, 0);
    for (int i = 1; i <= n; ++i) {
        cin >> a[i];
        prefix[i] = prefix[i-1] + a[i];
    }
    ll res = 0;
    for (int l = 1; l <= n; ++l) {
        for (int r = l; r <= n; ++r) {
            ll sum = prefix[r] - prefix[l-1];
            if (abs(sum) > S) res++;
        }
    }
    cout << res << endl;
    return 0;
}
\end{lstlisting}
\end{document}
% \textbf{Đề bài:}

% \textbf{Yêu cầu:}

% \textbf{Dữ liệu vào:}

% \textbf{Dữ liệu ra:}

% \textbf{Ràng buộc dữ liệu:}

% \begin{itemize}

% \end{itemize}


% \textbf{Ví dụ:}
% \begin{tcolorbox}[colback=gray!5!white, colframe=blue!50!black, title=Input]
% 3\\
% 1\\
% 2\\
% 3
% \end{tcolorbox}
% \begin{tcolorbox}[colback=gray!5!white, colframe=green!50!black, title=Output]
% 6\\
% 30\\
% 84
% \end{tcolorbox}
